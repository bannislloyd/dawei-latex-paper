

%\documentclass[pra,twocolumn,epsfig,rotate,superscriptaddress,showpacs]{revtex4}


% \mathcal{u}se only LaTeX2e, calling the article.cls class and 12-point type.

\documentclass[prl,onecolumn]{revtex4-1}
\usepackage{graphicx}
\usepackage{epsfig}
\usepackage{epsf}
\usepackage{amssymb}
\usepackage{amsmath}
\usepackage{amsthm}
\usepackage{multirow}
\usepackage{hyperref}

\usepackage[framed,numbered]{matlab-prettifier}
\let\ph\snippetPlaceholder
\lstset
{
  style = Matlab-editor,
  escapechar      = ",
}

\renewcommand{\familydefault}{\sfdefault}  %% arial font
% \usepackage{times} %% times new roman font

\newcommand{\bra}[1]{\langle #1|}
\newcommand{\ket}[1]{|#1\rangle}
\newcommand{\dir}{$\backslash$}
\newcommand{\be}{\begin{equation}}
\newcommand{\ee}{\end{equation}}
\newcommand{\bea}{\begin{eqnarray}}
\newcommand{\eea}{\end{eqnarray}}
\newcommand{\Fig}[1]{Fig.\,\ref{#1}}
\newcommand{\Eq}[1]{Eq.\,(\ref{#1})}
\newcommand{\la}{\langle}
\newcommand{\ra}{\rangle}
\newcommand{\nl}{\nonumber \\}
%\usepackage[usenames]{color}
%\definecolor{Red}{rgb}{1,0,0}
%\definecolor{Blue}{rgb}{0,0,1}




\setlength{\parindent}{0pt} % no incident
%\renewcommand{\baselinestretch}{1.2} % space between two lines
\setlength{\parskip}{6\lineskip} % space between two paragraphs

%%%%%%%%%%%%%%%%% END OF PREAMBLE %%%%%%%%%%%%%%%%



\begin{document}

% Include your paper's title here
\title{Code updating history of Jump Code Experiment}
\author{Dawei Lu}

\begin{abstract}
Notes about the Matlab code used in the Jump Code Experiments.
\end{abstract}
\today

\maketitle

\section{Dec 15, 2014}

All the codes are available in SVN 'https:\dir \dir dawei-qip-matlab.googlecode.com\dir svn\dir trunk\dir Jump Code'

First consider the classical case. Without correction, one needs one bit with the two states $\ket{+}$ and $\ket{-}$.
The channels and the related unitary are  

\begin{lstlisting}
% detected jump channel
A0 = [1,0;0,sqrt(1-r)]; A1 = [0,0;0,sqrt(r)];

% unitary operator
U = [1,0,0,0; 0, sqrt(1-r),0,sqrt(r);0,0,-1,0;0,sqrt(r),0,-sqrt(1-r)];
\end{lstlisting}



\newpage




\begin{thebibliography}{99}
%\bibitem{Moussa2012} O. Moussa, M. da Silva, C. Ryan, and R. Laflamme, Phys. Rev. Lett. \textbf{109}, 070504 (2012).

\end{thebibliography}


\end{document}
