\documentclass[aps,12pt,nofootinbib,superscriptaddress,longbibliography,
showpacs]{revtex4-1}

\usepackage[utf8]{inputenc}
\usepackage[T1]{fontenc}
\usepackage[english]{babel}  % english language

\usepackage{amssymb,amsmath,amsfonts,amsthm}
\usepackage{mathtools}
\usepackage{verbatim}
\usepackage{enumerate}
%\usepackage{tensor}
\usepackage{fancyhdr}
\pagestyle{fancy}
\usepackage{graphicx}
\graphicspath{{pics/}}
\usepackage{hyperref}
\usepackage{bbm}


\newcommand{\bydef}{\stackrel{\mathrm{def}}{=}}
\renewcommand{\baselinestretch}{1}  % cancels any possible double spacing

\usepackage{color}
%Some shortcuts added for edits
\definecolor{dred}{rgb}{.8,0.2,.2}
\definecolor{ddred}{rgb}{.8,0.5,.5}
\definecolor{dblue}{rgb}{.2,0.2,.8}
% suggested change
\newcommand{\add}[1]{\textcolor{dred}{*** #1 ***}} 
% suggested to remove
\newcommand{\out}[1]{\textcolor{ddred}{\textbf{[}\emph{#1}\textbf{]}}}
% comment or remark
\newcommand{\yo}[1]{\textcolor{dblue}{\textbf{[}#1\textbf{]}}}
\newcommand{\todo}[1]{\textbf{\underline{\textcolor{dblue}{\textbf{[}#1\textbf{]}}}}}

\theoremstyle{plain}
\newtheorem{theorem}{Theorem}   %[section]
\newtheorem{proposition}[theorem]{Proposition}
\newtheorem{lemma}[theorem]{Lemma}
\newtheorem{corollary}[theorem]{Corollary}

\theoremstyle{definition}
\newtheorem{definition}[theorem]{Definition}
\newtheorem{example}[theorem]{Example}
\newtheorem{remark}[theorem]{Remark}

% bra and ket: 
\newcommand{\bra}[1]{\mbox{$\langle #1|$}}
\newcommand{\ket}[1]{\ensuremath{|#1\rangle}}
\newcommand{\braket}[2]{\mbox{$\langle #1|#2\rangle$}}
\newcommand{\ketbra}[2]{\mbox{$|#1\rangle\langle #2|$}}
\newcommand{\iprod}[2]{\ensuremath{\langle #1,#2 \rangle}}

\newcommand{\gate}[1]{\ensuremath{\text{\sc #1}}}
\newcommand{\COPY}[1][]{\ensuremath{\gate{COPY}_{#1}}}

\newcommand{\comm}[2]{\ensuremath{\left[#1, #2\right]}}

\newcommand{\eq}{\Leftrightarrow}

\DeclareMathOperator{\Tr}{Tr}
\DeclareMathOperator{\Real}{Re}
\DeclareMathOperator{\Imag}{Im}
\DeclareMathOperator{\Span}{span}
\DeclareMathOperator{\diag}{diag}
\DeclareMathOperator{\Aut}{Aut} % automorphism group
\DeclareMathOperator{\End}{End} % set of endomorphisms

\newcommand{\projector}[1]{\mbox{$|#1\rangle\langle #1|$}}

\newcommand{\x}{\mathbf{x}}
\newcommand{\y}{\mathbf{y}}

\newcommand{\hprod}{\odot}

\newcommand{\I}{\openone}     % identity operator
\newcommand{\R}{{\mathbb R}}  % real numbers
\newcommand{\C}{{\mathbb C}}  % complex numbers
\newcommand{\hilb}[1]{\ensuremath{\mathcal{#1}}} % Hilbert space
\newcommand{\swap}{{\sf{SWAP}}}
\newcommand{\ie}{i.e.}

% defines logic function names, to look nice 
\newcommand{\be}{\begin{equation}}
\newcommand{\ee}{\end{equation}}

\newcommand{\Figref}[1]{Figure \ref{#1}}

% Quantum communication, Formalism

\def\thesection{%
\arabic{section}}%
\def\thesubsection{%
\arabic{subsection}}%
\def\thesubsubsection{%
\arabic{subsubsection}}%
\def\theparagraph{%
\arabic{paragraph}}%
\def\thesubparagraph{%
\theparagraph.\arabic{subparagraph}}%
\setcounter{secnumdepth}{5}%

% please leave these here 


% fonts 
\def\1#1{{\bf #1}}
\def\2#1{{\cal #1}}
\def\7#1{{\mathbb #1}}


%-------------------------------------------
% Tomi's stuff
%-------------------------------------------

% Fractions
\newcommand{\half}{\mbox{$\textstyle \frac{1}{2}$}}
\newcommand{\quarter}{\mbox{$\textstyle \frac{1}{4}$}}

% Dirac notation
%\newcommand{\ket}[1]{\left | \, #1 \right \rangle}
\newcommand{\kets}[1]{ | \, #1 \rangle}
%\newcommand{\bra}[1]{\left \langle #1 \, \right |}
\newcommand{\bras}[1]{ \langle #1 \, \right}
%\newcommand{\braket}[2]{\left\langle\, #1\,|\,#2\,\right\rangle}
\newcommand{\brakets}[2]{\langle\, #1\,|\,#2\,\rangle}
\newcommand{\bracket}[3]{\left\langle #1 \left| #2 \right| #3 \right\rangle}
\newcommand{\brackets}[3]{\langle #1 | #2 | #3 \rangle}
\newcommand{\proj}[1]{\ket{#1}\bra{#1}}
\newcommand{\av}[1]{\langle #1\rangle}
\newcommand{\outprod}[2]{\ket{#1}\bra{#2}}

% Common operators
\newcommand{\tr}{\textrm{tr}}
%\newcommand{\Tr}{\mathrm{Tr}}
\newcommand{\od}[2]{\frac{\mathrm{d} #1}{\mathrm{d} #2}}
\newcommand{\pd}[2]{\frac{\partial #1}{\partial #2}}
\newcommand{\dt}[1]{\frac{\partial #1}{\partial t}}

% Second quantisation
\newcommand{\an}[1]{\hat{#1}}
\newcommand{\cre}[1]{\hat{#1}^\dag}
\newcommand{\vac}{\ket{\textrm{vac}}}

% Other quantum
\newcommand{\cc}{\textrm{c.c.}}

% Common letters
%\newcommand{\ee}{\mathrm{e}}
%\newcommand{\ii}{\mathrm{i}}
%\newcommand{\dd}{\mathrm{d}}
\newcommand{\identity}{\mathbbm{1}}

% Common symbols
\newcommand{\up}{\uparrow}
\newcommand{\down}{\downarrow}

\renewcommand{\Re}{\mathfrak{Re}}
\renewcommand{\Im}{\mathfrak{Im}}

% Letters in different styles

\newcommand{\AAA}{\mathbf{A}}
\renewcommand{\AA}{\mathcal{A}}
\newcommand{\aaa}{\mathbf{a}}
\renewcommand{\aa}{\mathrm{a}}
\newcommand{\BBB}{\mathbf{B}}
\newcommand{\BB}{\mathcal{B}}
\newcommand{\bbb}{\mathbf{b}}
\newcommand{\bb}{\mathrm{b}}
\newcommand{\CCC}{\mathbf{C}}
\newcommand{\CC}{\mathcal{C}}
\newcommand{\ccc}{\mathbf{c}}
\renewcommand{\cc}{\mathrm{c}}
\newcommand{\DDD}{\mathbf{D}}
\newcommand{\DD}{\mathcal{D}}
\newcommand{\ddd}{\mathbf{d}}
\newcommand{\dd}{\mathrm{d}}
\newcommand{\EEE}{\mathbf{E}}
\newcommand{\EE}{\mathcal{E}}
\newcommand{\eee}{\mathrm{e}}
%\newcommand{\ee}{\mathrm{e}}
\newcommand{\FFF}{\mathbf{F}}
\newcommand{\FF}{\mathcal{F}}
\newcommand{\fff}{\mathbf{f}}
\newcommand{\ff}{\mathrm{f}}
\newcommand{\GGG}{\mathbf{G}}
\newcommand{\GG}{\mathcal{G}}
\renewcommand{\ggg}{\mathbf{g}}
\renewcommand{\gg}{\mathrm{g}}
\newcommand{\HHH}{\mathbf{H}}
\newcommand{\HH}{\mathcal{H}}
\newcommand{\hhh}{\mathbf{h}}
\newcommand{\hh}{\mathrm{h}}
\newcommand{\III}{\mathbf{I}}
\newcommand{\II}{\mathcal{I}}
\newcommand{\iii}{\mathbf{i}}
\newcommand{\ii}{\mathrm{i}}
\newcommand{\JJJ}{\mathbf{J}}
\newcommand{\JJ}{\mathcal{J}}
\newcommand{\jjj}{\mathbf{j}}
\newcommand{\jj}{\mathrm{j}}
\newcommand{\KKK}{\mathbf{K}}
\newcommand{\KK}{\mathcal{K}}
\newcommand{\kkk}{\mathbf{k}}
\newcommand{\kk}{\mathrm{k}}
\newcommand{\LLL}{\mathbf{L}}
\newcommand{\LL}{\mathcal{L}}
\renewcommand{\lll}{\mathbf{l}}
\renewcommand{\ll}{\mathrm{l}}
\newcommand{\MMM}{\mathbf{M}}
\newcommand{\MM}{\mathcal{M}}
\newcommand{\mmm}{\mathbf{m}}
\newcommand{\mm}{\mathrm{m}}
\newcommand{\NNN}{\mathbf{N}}
\newcommand{\NN}{\mathcal{N}}
\newcommand{\nnn}{\mathbf{n}}
\newcommand{\nn}{\mathrm{n}}
\newcommand{\OOO}{\mathbf{O}}
\newcommand{\OO}{\mathcal{O}}
\newcommand{\ooo}{\mathbf{o}}
\newcommand{\oo}{\mathrm{o}}
\newcommand{\PPP}{\mathbf{P}}
\newcommand{\PP}{\mathcal{P}}
\newcommand{\ppp}{\mathbf{p}}
\newcommand{\pp}{\mathrm{p}}
\newcommand{\QQQ}{\mathbf{Q}}
\newcommand{\QQ}{\mathcal{Q}}
\newcommand{\qqq}{\mathbf{q}}
\newcommand{\qq}{\mathrm{q}}
\newcommand{\RRR}{\mathbf{R}}
\newcommand{\RR}{\mathcal{R}}
\newcommand{\rrr}{\mathbf{r}}
\newcommand{\rr}{\mathrm{r}}
\newcommand{\SSS}{\mathbf{S}}
\renewcommand{\SS}{\mathcal{S}}
\newcommand{\sss}{\mathbf{s}}
\renewcommand{\ss}{\mathrm{s}}
\newcommand{\TTT}{\mathbf{T}}
\newcommand{\TT}{\mathcal{T}}
\newcommand{\ttt}{\mathbf{t}}
\renewcommand{\tt}{\mathrm{t}}
\newcommand{\UUU}{\mathbf{U}}
\newcommand{\UU}{\mathcal{U}}
\newcommand{\uuu}{\mathbf{u}}
\newcommand{\uu}{\mathrm{u}}
\newcommand{\VVV}{\mathbf{V}}
\newcommand{\VV}{\mathcal{V}}
\newcommand{\vvv}{\mathbf{v}}
\newcommand{\vv}{\mathrm{v}}
\newcommand{\WWW}{\mathbf{W}}
\newcommand{\WW}{\mathcal{W}}
\newcommand{\www}{\mathbf{w}}
\newcommand{\ww}{\mathrm{w}}
\newcommand{\XXX}{\mathbf{X}}
\newcommand{\XX}{\mathcal{X}}
\newcommand{\xxx}{\mathbf{x}}
\newcommand{\xx}{\mathrm{x}}
\newcommand{\YYY}{\mathbf{Y}}
\newcommand{\YY}{\mathcal{Y}}
\newcommand{\yyy}{\mathbf{y}}
\newcommand{\yy}{\mathrm{y}}
\newcommand{\ZZZ}{\mathbf{ZZ}}
\newcommand{\ZZ}{\mathcal{ZZ}}
\newcommand{\zzz}{\mathbf{z}}
\newcommand{\zz}{\mathrm{z}}


% Referencing
\newcommand{\eqr}[1]{Eq.~(\ref{#1})}
\newcommand{\fir}[1]{Fig.~\ref{#1}}
\newcommand{\secr}[1]{Sec.~\ref{#1}}
\newcommand{\apr}[1]{App.~\ref{#1}}
\newcommand{\chr}[1]{Ch.~\ref{#1}}


% kill double space 
\renewcommand{\baselinestretch}{1} 

% Complexity classes 
\newcommand{\NP}{\mathrm{NP}}
\renewcommand{\P}{\mathrm{P}} 

\begin{document}

%\newcommand{\captionfonts}{\small}
%\makeatletter  % Allow the use of @ in command names
%\long\def\@makecaption#1#2{%
%  \vskip\abovecaptionskip
%  \sbox\@tempboxa{{\captionfonts #1: #2}}%
%  \ifdim \wd\@tempboxa >\hsize
%    {\captionfonts #1: #2\par}
%  \else
%    \hbox to\hsize{\hfil\box\@tempboxa\hfil}%
%  \fi
%  \vskip\belowcaptionskip}
%\makeatother   % Cancel the effect of \makeatletter

\title{Necessary and sufficient conditions for a quantum process to break
time-reversal symmetry}



\author{Ville Bergholm}
\affiliation{ISI Foundation,
Via Alassio 11/c, 10126
Torino, Italy}

\author{Mauro Faccin}
%\email{jacob.biamonte@qubit.org}
\affiliation{ISI Foundation,
Via Alassio 11/c, 10126
Torino, Italy} 

\author{Jacob Biamonte}
%\email{jacob.biamonte@qubit.org}
\affiliation{ISI Foundation,
Via Alassio 11/c, 10126
Torino, Italy}

\author{Tomi Johnson}
%\email{jacob.biamonte@qubit.org}
\affiliation{ISI Foundation,
Via Alassio 11/c, 10126 
Torino, Italy} 
\affiliation{University of Oxford} 

%\author{Sabre Kais?}
%\email{jacob.biamonte@qubit.org}
%\affiliation{Purdue}
%\affiliation{QEERI} 





\begin{abstract}
Here we're just trying to understand this stuff better.  
\end{abstract} 


\maketitle

\section{Time reversal symmetry} 

A quantum process is time-reversal symmetric iff the following quantity
vanishes for all time $t$.  

\begin{equation}
 |\bra{m}e^{-itH}\ket{n}|^2 - |\bra{m}e^{itH}\ket{n}|^2 = 0
\end{equation}

Which is equivalent to. 

\begin{equation}
 \bra{m}e^{-itH}\ket{n}\bra{n}e^{itH}\ket{m} -
\bra{m}e^{itH}\ket{n}\bra{n}e^{-itH}\ket{m} = 0
\end{equation}

We note the following identity, decomposing the propagator into two commuting
terms
\begin{equation}
 e^{-itH} = \text{cosh}(itH) - \text{sinh}(itH)
\end{equation}
the first is even and the second is odd in $t$.  From this identity, in terms of
components 

\begin{align}
 (\text{cosh}(itH)_{mn} -& \text{sinh}(itH)_{mn})(\text{cosh}(itH)_{nm} +
\text{sinh}(itH)_{nm})\\
-& (\text{cosh}(tH)_{mn} +
\text{sinh}(itH)_{mn})(\text{cosh}(itH)_{nm} - \text{sinh}(itH)_{nm}) = 0
\nonumber 
\end{align}

which reduces to two times the following quantity, which must vanish identically
for a process to be time reversal symmetric 

\begin{equation}\label{eqn:twist}
 \text{sinh}(itH)_{nm}\text{cosh}(itH)_{mn} -
\text{sinh}(itH)_{mn}\text{cosh}(itH)_{nm} = 0 
\end{equation}

The physical interpretation of this equation is relevant to us.  We will
consider graphs and focus on the number of edges in the possible paths
connecting two nodes, $n$ and $m$.  Stated roughly, $\text{sinh}(itH)_{nm}$
accounts for the odd transitions between $n$ and $m$ whereas
$\text{cosh}(itH)_{nm}$ accounts for the even ones.   In each case, the physical
interpretation is clear.  These even and odd transition terms 
account for all possible even or odd paths between $n$ and $m$, wherein paths
with higher length decrease in importance, with a penalty which is inverse
factorial in the path-length.  

First of all, if the path connecting nodes $n$ and $m$ has either exclusively an
even number of steps, or exclusively an odd number, then the quantity
\eqref{eqn:twist} surely vanishes.  This is always the case for a tree.  For an
example of a looped graph where the number of transitions between two nodes is
either always even or always odd consider $C_4$, a ring graph with four sites.
Here the paths connecting opposing nodes will
always be even, whereas those connecting neighboring ones is always odd,
causing either the sinh terms or the cosh terms to vanish exclusively and
hence, \eqref{eqn:twist} to vanish.  Such a quadrilateral network allows only
time symmetric evolutions, regardless of the intrinsic couplings.  

Second, in the absence of
phases (an intrinsic property of the connections), each of the two terms above
are identical, and so their difference vanishes and the process is necessarily
time-symmetric, regardless of the connecting geometry (which is an extrinsic
property, emerging from the global topology of the network). Thirdly, we can
note that for the quantity \eqref{eqn:twist} to be non-zero, it is necessary
that
there be both an even number of edges and also an odd number of edges
connecting $n$ and $m$.  In other words, there must be a loop
between $n$ and $m$, providing the existence of an even length as well as an odd
length path. Fourthly, there must be a phase difference in the transitions
experiences from a walk taking an even path from $n$ to $m$ times an odd path
from $m$ to $n$. The difference between this and its reverse must be
non-vanishing for the process to break time-symmetry.  


Finally we remark that it's not necessary to consider all possible paths
between two nodes as (a priori) purportedly prescribed by \eqref{eqn:twist}. 
For a process with $N$ nodes, the Cayley-Hamilton theorem asserts that there is
a vanishing polynomial with highest degree at most $N$. This means that
transitions of higher order can necessarily be expressed in terms of $N$-long
paths.  


\section{Questions}

\begin{enumerate} 
\item other ideas?   
\end{enumerate} 







\bibliography{chiral-bib}

\end{document}

 
