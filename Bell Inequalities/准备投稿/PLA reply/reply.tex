
\documentclass[a4paper,12pt]{article}


%++++++++++++++++++++++Packages++++++++++++++++++++++
\usepackage{ccmap}
%\usepackage{times}
%\usepackage{palatino}
\usepackage{newcent}
%\usepackage{bookman}
\usepackage{dingbat}
\usepackage{pifont}
\usepackage{dsfont}
\usepackage{bbding}
\usepackage{bbm}
\usepackage{bm}
\usepackage{mathbbold}
\usepackage{mathrsfs}
\usepackage[tbtags]{amsmath}
\usepackage{amssymb}
\usepackage{amsthm}
\usepackage{amsfonts}
\usepackage{graphicx}
\usepackage{xcolor}
\usepackage{fancybox}
\usepackage{booktabs}
\usepackage{tabularx}
\usepackage[dvipdfmx]{hyperref}
\hypersetup{colorlinks=true,breaklinks=true,pdfstartview=FitH,linkcolor=blue}
\usepackage{indentfirst}

%+++++++++++++ҳ���ʽ++++++++++++++++++++


\renewcommand{\baselinestretch}{1.66}
\renewcommand{\arraystretch}{1.5}
%\linespread{1.5}
\voffset   -2cm%
\addtolength{\textheight}{3cm}%
\addtolength{\hoffset}{-1.5cm}%
\addtolength{\textwidth}{4cm}
%\footskip=30pt
%\hoffset  -1.25cm



%++++++++++++++++++++++++++++++++++++++++
\renewcommand{\textfraction}{0.15}
\renewcommand{\topfraction}{0.85}
\renewcommand{\bottomfraction}{0.65}
\renewcommand{\floatpagefraction}{0.60}




%---------------------------------------------------------------------------------
\newcommand{\bra}[1]{\langle #1|}
\newcommand{\ket}[1]{|#1\rangle}
\newcommand{\zuo}{\langle}
\newcommand{\you}{\rangle}
\newcommand{\kai}{\CJKfamily{kai}}
\newcommand{\hei}{\CJKfamily{hei}}
\newcommand{\fs}{\CJKfamily{fs}}

%---------------------------------------------------------------------------------

\makeatletter
\def\equalsfill{$\m@th\mathord=\mkern-7mu
\cleaders\hbox{$\!\mathord=\!$}\hfill \mkern-7mu\mathord=$}
\makeatother

\makeatletter
\def\@makefnmark{\hbox{\textsuperscript{\textcolor[rgb]{1.00,0.00,0.00}{(\hspace{2pt}\@thefnmark{}\hspace{1pt})}}}} \makeatother
%========================================================================================

\begin{document}

\noindent We are very grateful to the referees for his/her pertinent
comments and useful suggestions which indeed help us to improve the
quality of our paper. According to referees' suggestions, we have
revised our manuscript.

\section{ \textit{Reply to Referee}}
\subsection{\textit{Reply to Reviewer 1}}
We thank referee for his/her helpful remarks.


\emph{\textbf{Question:}}  The Authors' use of the term
"experimental" seems misleading; as far as I could see, they merely
performed "computer experiments." There are also many typos.

\emph{\textbf{Our reply:}}
 Our work is real NMR experiments similar to [ \emph{New. J. Phys. \textbf{10}, 033020 (2008)} ], but not
  ``computer experiments". In [ \emph{New. J. Phys. \textbf{10}, 033020 (2008)}], Souza \emph{et
al} implemented an experiment to simulate the violation of CHSH
inequality in a 2-qubit NMR system. And we implemented such a
simulation in 3-qubit NMR system. The difference is Souza \emph{et
al} only implemented on 2-qubit Bell state, but we carried out the
simulation of two different Bell-type inequalities for nonmaximal
entangled states (3-qubit entangled states). Our experimental data
were shown by the blue squares in Fig.3 and Fig.4 in the text.


As to typos, we apologized to make you misread caused by them. We
have tried our best to revise the typos of our manuscript and make
it more easily read. Furthermore, we also found a native English
speaker to check our manuscript.

\subsection{\textit{Reply to Reviewer 2}}

We thank referee considered that our work is certainly worthwhile
and appears to have been successful. Furthermore, We also appreciate
referee for his/her helpful remarks which make us avoid some
confusion. We give the explanations about referee's questions.

[1]. \emph{\textbf{Question:}} The main confusion for me arises from
the usage of "simulation" both in the title and content of the
paper. Is the inequality violated or not? Is it only a simulation?
Or is this word used because the state is highly mixed? Or is it
used because there can be no violation of Local Realism in the
experiment?


\emph{\textbf{Our reply:}} Our work is real NMR experiments to
simulate the violation of Bell-type inequalities similar to [
\emph{New. J. Phys. \textbf{10}, 033020 (2008)}]. As mentioned in [
\emph{New. J. Phys. \textbf{10}, 033020 (2008)}], the reason that we
used the word ``simulation" is that these experiments were carried
out in NMR system. The initial state in NMR is prepared from the
thermal equilibrium into a highly mixed state called pseudo-pure
state (PPS)[\emph{\emph{Proc. R. Soc. Lond. A} 454, 447 (1998)}]
\begin{equation}\label{PPS}
\rho_{pps}=\frac{(1-\varepsilon)}{2^n}I_{2^n}+\varepsilon\left\vert
\psi\right\rangle\left\langle\psi\right\vert,
\end{equation}
where $\varepsilon\approx10^{-5}$ is the polarization at room
temperature. Braunstein \emph{et al} [ \emph{Phys. Rev. Lett. \textbf{83},
 1054 (1999)}] pointed out that any state
of the form \eqref{PPS} is not entangled whenever $\varepsilon\leq
\frac{1}{(1+2^{2n-1})}$. However, with respect to scale-independent
NMR observations and unitary evolution, a pseudopure state is
equivalent to the corresponding pure state [quant-ph/0207172 (2002)
]. NMR is only sensible for the deviation part of \eqref{PPS}, which
behaves like a ``pure entangled state". Hence, NMR experiment can
still exhibit quantum properties. Although NMR system is local, it
really implement a simulation of the violation of Bell-type can
inequalities, Souza \emph{et al} have simulated the violation of
2-qubit CHSH inequality on NMR [ \emph{New. J. Phys. \textbf{10},
033020 (2008)}]. At the same time, as to distinguish from the true
experiment of exhibiting non-local effects, such as in quantum
optics \emph{et al}, we use the word "simulation".

[2]. \emph{\textbf{Question:}}  These latter two questions also seem
to be intermixed in the conclusions; it is unclear, to say the
least. The mixed part has in principle nothing to do with the
question of local realism, it is quite a different issue. Many of
the conclusions at the end are questionable and not supported by
theory or experiment in the rest of the paper.

\emph{\textbf{Our reply:}}  The referee's view,``The mixed part has
in principle nothing to do with the question of local realism", is
correct. We realized our conclusion, `` That is to say, our results
are also consistent with the classical theory, depending on whether
we have considered the mixed part $I_{2^n}$", is not strict. In
order to avoid misunderstanding and make the conclusion clear, we
delete this sentence .


[3]. \emph{\textbf{Question:}}  The inequalities need to be better
specified: what does the comma stand for in E(A1,B2,C2) (equation
(2) and others)? Multiplication? Usually these entities are
expectation values of a product of experimental outcomes, but this
is completely unclear in the present paper.

\emph{\textbf{Our reply:}}  Following the referee's suggestion, we
give a clear specified about inequalities in the revised text.

[4]. \emph{\textbf{Question:}}  There are other things that are
unclear, like what "several sets of observers" signify just below
equation (7). Or what "sorts of unexpected properties" have been
found recently at the start of section III. There are also lots of
language errors of different kinds.

\emph{\textbf{Our reply:}} We have tried our best to revise the
language errors of our manuscript and make it more easily read. Such
as, we replaced the sentence below Eq.(8) `` We took several sets of
observers to do..." as `` We took the observers mentioned above
($\bm{\sigma_{n_{1}}}$, $\bm{\sigma_{n_{2}}}$) to do... ". Another
example, the sentence `` Recently, sorts of unexpected properties
about non-maximal entangled states have been shown " has been
replaced as `` Recently, much work about non-maximal entangled
states has been done ".

We have also changed many sentences and words to conform to the need
of document in the journal pages.

\section{ \textit{List of the Changes}}

[1]. We added the equation $E(A_{i},B_{j},C_{k})=\langle
A_{i}B_{j}C_{k}\rangle_{avg}$ above the original Eq.(2) in section
II.

[2]. We replaced the sentence below Eq.(8) `` We took several sets
of observers to do..." as `` We took the observers mentioned above
($\bm{\sigma_{n_{1}}}$, $\bm{\sigma_{n_{2}}}$) to do... ".

[3]. In the beginning of Section III, the sentence `` Recently,
sorts of unexpected properties about non-maximal entangled states
have been shown " has replaced as `` Recently, much work about
non-maximal entangled states have been done ".

[4]. In the second paragraph of Section V, the sentence `` That is
to say, our results are also consistent with the classical theory,
depending on whether we have considered the mixed part $I_{2^n}$"
was deleted.

[5] . In the end of second paragraph of Section II, we added such
sentence `` In other words, the two dichotomic observables allowed
to be chosen for A,B,C are $\bm{\sigma_{n_{1}}}$ and
$\bm{\sigma_{n_{2}}}$. ".

[6] . We have also changed a lot of typos, the style of writing in
many sentences and words to conform to the need of document in the
journal pages.

We sincerely hope that the revised version will now be acceptable
for publication in Physics Letters A.

\bigskip
Sincerely yours,

Ren Changliang









\end{document}

%==========================================================================


����һ������ɫ�ĺ���, Ϊempheq����
%--------------------------------------------------------------------------
\definecolor{myblue}{rgb}{.8, .8, 1}
\newcommand*\mybluebox[1]{%
\colorbox{myblue}{\hspace{1em}#1\hspace{1em}}}
%--------------------------------------------------------------------------
\begin{empheq}[box=\mybluebox]{align}
Equation
\end{empheq}
