
\chapter{电子顺磁共振谱仪}
本章节介绍电子顺磁共振谱仪硬件,分为电磁场系统,微波桥系统,共振腔和调制场系统,探测系统和控制软件与数据处理。这里我们从一般的电子顺磁共振谱仪入手,着重阐述各个部件的功能和它们之间如何有机结合起来。

    \section{电磁场系统}
    电磁场系统被用来提供一个作用于样品的外加磁场。这个电磁场系统必须具备提供恒稳静磁场和线性扫场的功能。 在某些情况下,也需要提供低频调制场的输出功能。为了实现稳场功能,一般需要加入反馈控制。
        \begin{figure}[htbp]
        \begin{center}
          \includegraphics[width= 0.8\columnwidth]{figures/EMS.eps}
          \caption{电磁场系统简明框架图
          }
          \label{EMS}
        \end{center}
        \end{figure}

    图\ref{EMS}为电磁场系统的简明框架图。电磁场是由一对亥姆霍兹线圈和一个直流电源提供。当直流电流通过亥姆霍兹线圈的时候,在两对线圈中间一定区域会形成均匀的磁场,磁场$H$方向平行于两个线圈中心轴,磁场的强度在一定区域和线圈内部
    的电流大小成正比。实际使用时,由于电流通过亥姆霍兹线圈时会发热,电源本身输出也在一定范围内不稳定,因此,要实现稳定的静磁场输出,就需要引入反馈系统,来实时控制磁场。图\ref{EMS}中霍尔片,模数转换器,电脑,和数模转换器就构成了
    一个简单的反馈系统。如图\ref{Hall-effect}(图片作者S. P. Pedersen)子图(A)所示,当霍尔片($2$)置于磁场($3$)中时,磁场方向($4$)如图所示,我们使用电源($5$)驱动一个恒稳电流,其载流子($1$)在运动时由于洛伦茨力作用,会偏向一边,从而在霍尔片的另一边形成电压差。当我们变换电流方向或者磁场方向,
    可以看到霍尔电压的变化(子图B,C,D)。
        \begin{figure}[htbp]
        \begin{center}
          \includegraphics[width= 0.5\columnwidth]{figures/Hall-effect.eps}
          \caption{霍尔效应示意图
          }
          \label{Hall-effect}
        \end{center}
        \end{figure}
    在一定条件下,霍尔电压和测量的磁场值成正比,我们可以通过一系列校对,从而实现用霍尔片测试磁场。霍尔电压可以用模数转换器将其转换成数字信号,被输入电脑进行处理,经过和我们预设的磁场值进行比较,其偏差被用于修正磁场值。
    直流电源如果是压控,即使用电压控制起电流输出,我们可以将校对后的磁场值通过数模转换成电压信号,从而控制亥姆霍兹线圈的电流值,达到修正磁场的目的。如果要线性扫场,那就可以输出线性变化的控制电压,实现电流的线性变化。要实现调制场,可以施加一个
    调制电压与电流源控制接口,可以实现低频的交流电流输出,从而可以实现调制场功能。
    \section{微波桥系统}
    微波桥系统的作用简而言之是产生激励信号和接受经过和样品相互作用后的激励信号,并且将样品的信息从信号中解析出来。
    为方便理解,本章节我们使用通讯(Telecommunications)中常用的概念:发送器(Transmitter),接收器(Receiver)来阐述微波桥的工作原理。
        \subsection{发送器}
        磁共振实验,我们需要一个激励信号来激励样品,从而获得样品的信息。发送器就起到了产生和调制激励信号的作用。对于连续波模式,通常发送器只需要输出一个可调功率的点频微波;对于脉冲模式,发送器还需要将输出调制成脉冲,通常是方波脉冲。

            \begin{figure}[htbp]
                \begin{center}
                  \includegraphics[width= 1\columnwidth]{figures/transmitter.eps}
                  \caption{发送器示意图
                  }
                  \label{transmitter}
                \end{center}
            \end{figure}

        图\ref{transmitter}是发送器的简明示意图。 其中微波源(元件$1$:VCO)是压控振荡器。一般我们使用耿氏二极管来做微波源,原因是此类固体元件价格便宜,性能稳定。压控振荡器的输出频率是和控制电压成单调关系(某个区域线性关系),因此,配合锁相环(Phase Lock Loop)的使用,可以输出高品质的微波,并且输出与主时钟(Mast Clock )同步。元件$2$为定向耦合器,用于进行功率分配。微波源输出的部分功率被用于锁相环,一部分被传送至接收器,作为解调制用,其余功率经过放大器(元件$3$)放大之后,被微波开关(元件$4$)调制成方波脉冲后,进入喇叭(元件$5$)被发送出去。在Q波段以下的谱仪中,我们通常使用共振腔来代替喇叭。在高波段谱仪中,
        喇叭被较多的使用\cite{3mm_ESR, 1mm_ESR}。共振腔的部分,我们会在后面章节讨论。

        \subsection{接收器}
        当微波与样品相互作用后,样品的信号被加载在微波上。通常来说,微波的频率比较高,例如工作在X波段的谱仪,典型的频率为$9.7~$GHz,试图直接解析出样品的信号是非常困难的。因此,接收器的主要作用是利用混频技术,将高频信号中的低频信号解析出来。
            \begin{figure}[htbp]
                \begin{center}
                    \includegraphics[width= 1\columnwidth]{figures/receiver.eps}
                    \caption{接收器示意图
                    }
                    \label{receiver}
                \end{center}
            \end{figure}
        图\ref{receiver}是接收器的简要示意图。喇叭将经过样品的信号收集起来,经过一个保护开关后被一个低噪声放大器(元件$5$)进行信号放大。这里我们选取噪声系数比较小的低噪声放大器是出于提高探测信噪比的考量。由于该放大器是
        信号采集经历的第一个放大器,其性能很大程度上决定了整个谱仪的探测灵敏度或者说信噪比。根据Friss公式
            \begin{equation}
            \label{Friss-equation}
            F_{total} = F_1 + \frac{F_2 - 1}{G_1} + \frac{F_3-1}{G_1G_2} + \frac{F_4 - 1}{G_1G_2G_3} +  \cdot\cdot\cdot
            \end{equation}
        其中$G_{i}$和$F_{i}$($i=1,2,3...$)表示级联的元件的增益和噪声因子(Noise factor)。如果元件是被动元件,譬如衰减器,那么其噪声系数数值上就是其衰减$dB$数(注意这里噪声系数是噪声因子的分贝表示)。对于我们的系统而言,加入暂不考虑保护开关的插损的影响,可以得到整个接收器的噪声因子为
            \begin{equation}
            \label{receiver-noise}
            F_{receiver} = F_{LNA}+ \frac{F_{rst}}{G_{LNA}}
            \end{equation}
        由上式可以看出,低噪声放大器对于接收器的信噪比有着重要的影响。

        信号经过低噪声放大器放大之后,进入一个混频器。混频器的作用是将信号与波源出来的参考信号进行比较,其实是一个解调制的过程,这样加载在高频微波上的低频信息(譬如样品在共振条件下对微波的吸收信息或者样品被高功率脉冲脉冲激发后的自由感应衰减信号)
        可以被解调制出来,然后经过一个$DC\sim500~MHz$的视频放大器的再次放大后,被模数转换器转换成数字信号,最后的结果被计算机记录并且分析。混频器通常选取正交混频器(IQ mixer)。输出的两路相位差九十度,因此可以在脉冲式电子顺磁共振实验中用来实现正交检测的功能。
        连续波模式下,出于提高信噪比的考虑,我们沿着静磁场方向增加了一个在$10\sim100~kHz$的调制磁场,所以在视频放大器之后又增加了一个锁相放大器,
        进行第二步的解调制,最后得到连续波谱。
    \section{共振腔和调制场系统}
        电子顺磁共振实验中,共振腔有着重要的作用。样品被放置于共振腔中,发送器产生的微波被输送至共振腔内,与样品相互作用之后被传输出至接收器。
        共振腔设计的要求是:
            \begin{enumerate}
              \item 具有合适的品质因子Q
              \item 磁场$B_1$方向与外加静磁场方向垂直
              \item 样品所在区域有着尽可能高和均匀的$B_1$场(尽可能低的电场)
            \end{enumerate}
        由于不同实验的要求,发展了不同形式的共振腔,譬如有裂环式共振腔(Loop-gap resonator),矩形共振腔(Rectanguler resonator),圆柱形共振腔(Cylindrical resonator)等。 在高频谱仪中,通常使用法布里-珀罗共振腔(Fabry–Perot Resonator)。裂环式共振腔有着较好的填充因子,并且品质因子可以控制在几十到一百之间,所以比较适合用于脉冲式电子顺磁共振实验中。矩形共振腔有着比较高的品质因子,所以被用于连续波电子顺磁共振实验,以期获得良好的信噪比。
        圆柱形共振腔在低温实验中有着良好的应用。
        
        
        我们以矩形谐振腔为例,介绍共振腔的设计。矩形谐振腔通常采用$TE_{102}$模式,其共振腔示意图和电磁场分布见图\ref{rectangular-resonator}
            \begin{figure}[htbp]
                \begin{center}
                    \includegraphics[width= 1\columnwidth]{figures/rectangular-resonator.eps}
                    \caption{矩形谐振腔及其电磁场分布示意图\cite{Weil1}
                    }
                    \label{rectangular-resonator}
                \end{center}
            \end{figure}
        图\ref{rectangular-resonator}.a是一个简易的矩形共振腔结构图,被用于反射式电子顺磁共振谱仪。外加静磁场沿着z轴,样品沿着y轴被放置于共振腔里面。微波能量的输入和输出都通过波导管,波导管和共振腔之间的耦合可以通过虹膜(Iris)来调节。
        虹膜是一个绝缘材料制成的螺丝,螺丝顶部安装一片金属。通过旋进和旋出螺丝来调节波导管和共振腔的耦合。图\ref{rectangular-resonator}.b和\ref{rectangular-resonator}.c为矩形共振腔里面电磁场分布示意图。
        子图(b)是电场分布,可以看到在理想情况下,电场在样品管位置的强度达到最小。子图(c)是磁场分布,在样品管区域,磁场方向和外加静磁场垂直,并且强度最大。
        
        
        调制场是由一对亥姆霍兹线圈组成。调制场的方向与静磁场的方向保持平行,产生幅度从毫高斯到一百高斯的交流电磁场,其频率从$10~$kHz到$100$kHz。我们将亥姆霍兹线圈等效成一个电感串联一个电阻,为了实现不同频率的调制场,我们使用了不同的电容与之串联,
        形成LCR电路。我们使用一个电流放大器来将波源产生的$10$kHz到$100~$kHz的交流电压信号转化成交流电流。当交流电流通过亥姆霍兹线圈后,就会在线圈中心区域产生正比于电流值大小的交流电磁场。调制场系统的等效电路见\ref{modulation-field-coil}
            \begin{figure}[htbp]
                \begin{center}
                    \includegraphics[width= 1\columnwidth]{figures/modulation-field-coil.eps}
                    \caption{调制场系统等效电路
                    }
                    \label{modulation-field-coil}
                \end{center}
            \end{figure}
        
        
        图\ref{modulation-field-coil}中交流电流源提供交流电流,保护电阻$R_p=20\Omega$用于保护电路中的电流不至于过大而对电路造成损伤,当交流电流的频率满足LCR电路的谐振条件$f=1/\sqrt{LC}$的时候,通过调制线圈的交流电$I_{ac}=V_{ac}/(R_p+R)$。我们可以通过监测
        保护电阻$R_p$两端的电压值来监测调制线圈中的调制电流大小,从而可以有效的控制调制场的大小。
    \section{探测系统}
    \section{控制软件与数据处理}




