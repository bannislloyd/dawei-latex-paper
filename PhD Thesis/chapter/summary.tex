
\chapter{总结与展望}

人类是宇宙中新的来访者,因此我们需要花费很长的时间和很多的精力来尝试理解宇宙。

在历史的长河中,每当我们得到一个发现,总会伴随着一个未解之谜促使我们去做更多的探索,甚至引发一场革命。

当古希腊的哲人把光看成是一粒粒非常小的光原子组成的粒子流时,阿尔·哈桑利用小孔成像实验证明了光是一种波;当胡克,惠更斯正在把光的波动说
发扬光大的时候,牛顿把粒子说和他的力学体系结合起来,从粒子的角度解释了薄膜透光,牛顿环和衍射实验;当牛顿的光粒子说如此地深入人心的时候,托马斯·杨又发现了
双缝干涉的波动现象;当波动说节节胜利之时,马吕斯又发现了光的偏振现象;当粒子说正要重整旗鼓之时,菲涅尔又发现了泊松亮斑,提出光是一种横波;当麦克斯韦在
此基础上发表了三篇不朽的电磁理论论文,赫兹又用实验证明了这套理论之时,殊不知赫兹的实验已经埋下了新革命的火种。

就在上个世纪初,新的谜团再次出现了。迈克尔逊-莫雷实验和黑体辐射又让当时的物理学的黄金时代上空蒙上了两朵乌云。
曾经的经典物理学大厦在金色光芒的辉映下,是那么的庄严雄伟,溢彩流光,令人不禁想起宙斯和众神
在奥林匹斯山上那亘古不变的宫殿。但谁又想到,这震撼人心的壮丽,却是斜阳投射在庞大帝国最后的余晖,这两朵乌云已呈铺天盖地之势,席卷整个物理学的
世界。

第一朵乌云,最终导致了相对论革命的爆发;而第二朵乌云,最终导致了量子论革命的爆发。

虽然直到今天,我们也很难理解薛定谔的猫为什么会同时存在死和活的状态;虽然我们也不相信相距很远的两个粒子
会有超距的纠缠和测量同时塌缩;虽然在1999年的剑桥牛顿研究所关于量子计算会议的投票上,哥本哈根和GRW都只有4票,而有30人选择了多世界和多历史
理论,甚至50多人选择无法抉择;但是

\emph{量子力学确实是,而且一直是“真实”的传记。}

更重要的是,量子力学的出现彻底改变了世界的面貌, 它比史上任何一种理论都引发了更多的技术革命。 核能、
计算机技术、新材料、能源技术、信息技术……这些都在根本上和量子理论密切相关。牵强一点说,
如果没有足够的关于弱相互作用力和晶体衍射的知识,DNA的双螺旋结构也就不会被发现,分子生物
学也就无法建立,也就没有如今这般火热的生物技术革命。再牵强一点说,没有量子力学,也就没
有欧洲粒子物理中心(CERN),而没有CERN,也就没有互联网的www服务,更没有划时代的网络革命。

结合上世纪另一个重要进展-信息科学的话,我们发现量子力学又到达另一个新的高度。费曼在1982年就注意到,当我们试图使用计算机来模拟某些物理过程,例如量子叠加的时候,计算量会随
模拟对象的增加而指数增长,以致使得经典的模拟很快变得不可能。费曼并未因此感到气
馁,相反,他敏锐地想到,也许我们的计算机可以使用实际的量子过程来模拟物理现象!如果说模
拟一个“叠加”需要很大的计算量的话, 为什么不用叠加本身去模拟它呢?每一个叠加都是一个不同的
计算,当所有这些计算都最终完成之后,我们再对它进行某种幺正运算,把一个最终我们需要的答
案投影到输出中去。

而这种猜想不仅理论上是正确的,实验上它也确实是可行的。当我们的处理器速度还只是满足摩尔定律每隔18个月翻倍的时候,
当政府、军队和银行还在给他们的大数增加一两位数,来保好几十年平安的时候,当分解一个250位的数字还需要全世界的经典计算机联网工作几百万年的时候,Shor算法横空出世,
我们只需要消耗几分钟的量子计算机时间就可以搞定这个问题。

即使我们花了十年从15到21再到143,即使实验上量子态的纠缠依然非常容易退相干,即使量子计算机的真正运用可能还要过好几十年,
但这个领域依然给我们描绘了非常诱人的图景,一切也都在向好的方向发展,即使以本论文观察,那些近几年相继完成的量子计算实验,
也证明我们确实一直在进步。

我偶尔还会怀念步入量子理论乃至量子计算领域之前那种因果一丝不苟,宇宙简单易懂的日子,正如《乱世佳人》
开头不无深情地说:“曾经有一片属于骑
士和棉花园的土地叫做老南方。在这个美丽的世界里,绅士们最后一次风度翩翩地行礼,骑士们最
后一次和漂亮的女伴们同行,人们最后一次见到主人和他们的奴隶。而如今这已经是一个只能从书
本中去寻找的旧梦,一个随风飘逝的文明。”即使偶尔会有这样的伤感,我依然还会歌颂北方扬基们
最后的胜利,因为我从他们那里得到更大的力量,更多的热情,还有未来在这条道路上走下去的更执着的信心。


 

