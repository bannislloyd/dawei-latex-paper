
\begin{thanks}

时光荏苒,一晃已经九年。很幸运来到中国科学技术大学,来到少年班,来到杜江峰老师的实验室。

在这个实验室,仿佛间我穿越到了战火纷飞,硝烟弥漫的三国时代。杜江峰老师率领的,正是那支扫平天下,威镇寰宇的虎狼之师-曹魏。

杜老师恰如曹孟德,运筹演谋,鞭挞宇内,揽申、商之法术,该韩、白之奇策。杜老师眼光极准,统筹极佳,作为统帅,具有超强的指挥才能,整个实验室在其调度之下
节节胜利,稳步前进;
同时,杜老师事必躬亲,几乎每日必亲临前线,与众将士同甘共苦,促使三军用命,各个奋勇争先;
而其用人惟贤,恩威并重。我勉强作为实验室发展壮大的见证人之一,在几年间亲眼目睹众多身怀抱负,志向远大的青年才俊前来投奔,并团结在其周围;生活中,杜老师虚怀若谷,
人格魅力出众,使得我们可以畅所欲言,经常迸出思维的火花。未来如有机会,势必再次感受魏武挥鞭之志。

彭新华老师则“天生郭奉孝,豪杰冠群英。腹内藏经史,胸中隐甲兵。”虽是后来才进入本实验室工作,但彭老师才识超群,足智多谋,出谋划策,功绩卓著。认识彭老师后,我才知以前所学之技术
仅为NMR中的皮毛,而限于自己才疏学浅,我毕业之时所掌握的NMR技术可能也仅是彭老师技术实力的皮毛。我相信在彭老师的运筹与决策下,NMR部队在未来势必取得超越官渡之战的更辉煌的胜利。

任晓涛老师则有王佐之才,如同荀彧辅佐曹操一样,忠心陪伴杜老师左右,即便杜老师经常工作至深夜也毫无怨言,还常助其英文回复。在我眼中,任老师清秀通雅,更曾帮我修改申请表,推荐信等,至今感激不尽。

苏吉虎老师,典型的鬼才,正如贾文和。可惜其领域与我相去甚远,一直不敢开口讨教。石名俊老师与秦敢老师则是荀攸与
程昱,皆有经天纬地之才,只可惜我理论太差,见识太浅,终不可学得两位理论大师之一分。

邹平与朱晶师兄虽现已离开实验室,但两人技术超强,正如许褚和典韦冠绝三军。不仅我受益匪浅,NMR很多技术也是自两人流传至今。作为朱少曾经的小弟,
终不能超越师傅。孙敏师姐就像蔡琰,虽出自曹魏,但现已转行做金融,期待她也能如蔡文姬在诗歌领域一样,在金融界打出一片天地。

陈宏伟与荣星师兄现为杜老师的左臂右膀,如同元让、妙才二兄弟之于曹操。夏侯惇勤于和戎,治于内政,恰如宏伟师兄之大总管之名;夏侯渊千里袭人,所向披靡,恰如荣星师兄
之雷厉风行之性。居琛勇师兄则为曹仁,督领诸将,竭忠尽职。有居居师兄在则诸事无忧,如樊城有子孝则不惧关羽。

太祖建兹武功,时之良将,五子为先。王亚智勇双全,威风凛凛,若威震逍遥津之张辽;鹏飞勇猛果敢,常为先锋,若乌巢奇袭之乐进;
发展坚毅沉稳,常为督率,若汝南破黄巾之于禁;夜宵擅于料敌,巧变制胜,若街亭败马谡之张郃;老徐治军严明,清廉自守,若长驱助曹仁之徐晃。
有此良将,天下何愁不定。

此外,还有两个重要人物不得不提。大波哥虽已远在西安,然其司马仲达般深谋远虑之道依旧值得我学习和借鉴;昌亮虽已奔赴日本且在实验室时如曹洪一般并不
起眼,但我犹记当年一起看球聊天,意气风发的日子。这里祝两位一切安好。

其他众多师兄师弟中如不出意外,黄璞将如邓艾,为未来之领军人物。其余如蒋峰建、孙春晓、杨威、李松键、杨佳慧、李俊、谭一鹏、朱进贤、郑文强、高雁翔、冯鹏博、
孔熙、温旭杰、徐挽杰、雷超、罗志煌、聂新芳、周辉、郭学仪、周经纬、陈明、袁峰、许祥坤、潘健、王恒岩,以及曾经的本科师弟林毅恒、吴三丰等人,在此一并谢过,祝你们
一生平安,前程似锦。

感谢在我彷徨之时把我引入正途的叶国华老师,以及在谱仪操作上给予无私帮助的王雨松老师,感谢所有的帮助过我的人。

最后,感谢我的父母和家人,虽然我还是区区一个小兵,但我会如你们期待一般的成长,争取早日成为三军统帅。

当然,不能忘了亲爱的蓉蓉,这篇论文有你2/3的功劳,别问我另外1/3是谁的。

\vskip 18pt

\begin{flushright}

鲁大为

\today

\end{flushright}

\end{thanks}
