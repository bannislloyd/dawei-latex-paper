
\chapter{电子顺磁共振原理}
\label{chap:esr-theory}
本章绍电子顺磁共振基本原理。我们从静态自旋哈密顿量出发,讨论每一项的意义。

    \section{静态自旋哈密顿量}

    通常,考虑一个有效自旋为$S$顺磁粒子和$m$个核自旋$I$组成的体系处在外磁场$B_0$中,哈密顿量可$\mathcal{H}_0$以写成如下形式\cite{static-spin-H}:
        \begin{equation}
            \begin{split}
            \label{H0}
            \mathcal{H}_0   &= \mathcal{H}_{EZ} + \mathcal{H}_{ZFS} + H_{HF} + \mathcal{H}_{NZ} + \mathcal{H}_{NQ} + \mathcal{H}_{NN}\\
                            &= \beta_e B_0 g S/\hbar + S D S + \sum\limits_{k=1}^m S A_k I_k - \beta_n \sum\limits_{k = 1}^m g_{n,k} B_0 I_k/\hbar + \sum\limits_{I_k > 1/2} I_k P_k I_k
                                + \sum\limits_{i\neq k}I_i d^{i,k}I_k
            \end{split}
        \end{equation}
    其中$\mathcal{H}_{EZ}$是电子自旋塞曼项,表示了电子自旋与外磁场的耦合; $\mathcal{H}_{ZFS}$表示零场分裂;$\mathcal{H}_{HF}$表示超精细耦合项,是电子自旋与周围核自旋相互作用;
    $\mathcal{H}_{NZ}$是核自旋塞曼项,表示核自旋与外磁场的相互作用;$\mathcal{H}_{NQ}$是核电四极矩项;
    $\mathcal{H}_{NN}$是核自旋与核自旋之间的相互作用项。每一项典型的能量值在图\ref{energy_levels}表示出来\cite{Schweiger_principles}。

        \begin{figure}[htbp]
            \begin{center}
              \includegraphics[width= 1\columnwidth]{figures/energy_levels.eps}
              \caption{电子自旋与核自旋相互作用强度典型值
              }
              \label{energy_levels}
        \end{center}
        \end{figure}


    \subsection{电子自旋塞曼相互作用}
    电子自旋处于外磁场$B_0$中,自旋与磁场的相互作用可以用塞曼相互作用
        \begin{equation}
        \label{E-Zeeman}
        \mathcal{H}_{EZ} = \beta_e B_0 g S/\hbar
        \end{equation}
    来表示。对于电子自旋$S = 1/2$,电子自旋塞曼项是最主要的相互作用项。电子自旋处在复杂的环境中,$g$因子可以写成张量形式,其三个主轴分量写成$g_x$、$g_y$和$g_z$。
    当电子自旋周围的环境是各项同性,譬如满足立方对称性(立方晶系),那么$g_x=g_y=gz$;
    如果降低对称性,例如具有某种轴对称(譬如正方晶系),那么可以有$g_x=g_y=g_{}\perp$和$g_z=g_{\parallel}$;
    如果是正交晶系,那么$g_x\neq g_y\neq g_z$。$g$因子反映了电子自旋所处的环境,从而被认为是电子自旋共振实验谱中最重要的信息之一,或者说是“指纹”信息。
    通常的电子自旋共振实验采用的样品是由大量的分子组成,被放置在磁场中间,由于实际的磁场并不理想,所有磁场空间分布不均匀。因此,不同位置的电子自旋感受的外场会有一定的不同,共振频率因此具有细微的不同,这个效应将对谱线的不均匀展宽有贡献。
    在脉冲电子顺磁共振实验中,例如自由感应衰减实验,对退相干时间$T_2^*$也有贡献。
    \subsection{核自旋塞曼相互作用}
    核自旋塞曼相互作用
        \begin{equation}
        \label{N-Zeeman}
        \mathcal{H}_{NZ} = \beta_n B_0 g_n S/\hbar
        \end{equation}
    同样表示了核自旋与外磁场的耦合。通常的电子自旋共振实验中,核自旋的塞曼项不会具有显著的贡献。


    \subsection{超精细耦合}
    电子自旋与核自旋之间的耦合作用被称为超精细耦合。通常被表示为$A$并且有:
        \begin{equation}
        \label{hyperfine interaction}
        H_{HF} =  S A I,
        \end{equation}
    大部分电子顺磁共振实验考察的样品中,这类相互作用起到重要的作用。$A$通常能够给出电子自旋周围丰富的结构信息。如同$g$因子, $A$也可以写成张量形式。
    超精细耦合相互作用对电子自旋退相干有着重要的贡献。由于系综中电子自旋周围的核自旋的状态不尽相同,因此核自旋热库中核自旋对电子自旋施加的有效场也不同,这种现象也会贡献为连续波谱线的不均匀增宽,是电子自旋相干的衰减的重要原因。

    \subsection{零场分裂}
    当电子自旋$S>1/2$时,就会存在零场分裂项
        \begin{equation}
        \label{zero-field-splitting}
        \mathcal{H}_{ZFS} = S D S
        \end{equation}。
    这是因为在$S>1/2$的时候,电子的偶极-偶极相互作用会去除基态$2S+1$简并。这种相互作用是与外加磁场无关,因此被称为零场分裂。
    以金刚石色心为例,该色心有一个氮原子和近邻的一个空穴组成。该空穴中有六个电子,其中两个为未成对电子,表现为一个$S=1$的电子自旋,它的零场分裂项$D$是两个未成对电子自旋的偶极相互作用造成,在主轴坐标系下也可以写成
    $\H_D = D[S_z^2-\frac{1}{3}S(S+1)]+E(S_x^2-S_y^2)$。$D$的大小通常是$2.87~$GHz, $E$在纳米颗粒金刚石中通常不为零。

    \subsection{核电四极矩}
    当核自旋$I>1$的时候,也会有核自旋四极矩项
        \begin{equation}
        \label{NQ}
        \mathcal{H}_{NQ} = I P I。
        \end{equation}
    在掺杂氮原子的富勒烯样品中,我们可以观察到这个核四极矩\cite{Revisited-endor}。如图\ref{nuclear-Quadrupole}(摘自\cite{Revisited-endor})所示,氮自旋$I=1$,因此具有非零的核电四极矩,图中蓝色实线的分裂就来源于此。
        \begin{figure}[htbp]
            \begin{center}
              \includegraphics[width= 0.8\columnwidth]{nuclear-Quadrupole.eps}
              \caption{利用电子-核双共振观测到核四极矩
              }
              \label{nuclear-Quadrupole}
            \end{center}
        \end{figure}

    \subsection{核自旋与核自旋相互作用}
    核自旋与核自旋之间的相互作用
        \begin{equation}
        \mathcal{H}_{NN} = I d I
        \end{equation}
    通常在电子自旋共振实验中解析不出来。在核磁共振实验谱学中,这是一个重要相互作用项,能够给出丰富的结构信息。

    \section{电子自旋$1/2$的量子力学描述}
    这里我们用一个电子自旋$S=1/2$的量子力学描述为一个例子,结合其在量子计算中的应用,介绍脉冲式电子顺磁共振实验的基本理论。我们首先介绍如何将一个$S=1/2$的自旋编码成量子比特,如何描述这个量子比特的动力学演化;
    接着我们介绍旋转坐标系的应用;最后我们介绍平均哈密顿量理论。

        \subsection{量子比特-Qubit}
        $S=1/2$的电子自旋被放置于静磁场中,由于塞曼分裂,会具有自旋向上$|\uparrow\rangle$和$|\downarrow\rangle$两个本征态。我们可以将这个电子自旋编码成一个量子比特(Qubit)。$|\uparrow\rangle$($|\downarrow\rangle$)
        被编码成$|0\rangle$($|1\rangle$)。这个电子自旋也可以被看成一个布洛赫矢量(Bloch vector),那么通过操纵这个电子自旋,矢量处在一个布洛赫球面上任意一点\ref{bloch-sphere},这可以被表达为
            \begin{equation}
            \label{bloch-vector}
            |\psi\rangle = \cos (\theta/2)|0\rangle + e^{i\phi}\sin (\theta/2)|1\rangle 。
            \end{equation}
        图表示一个矢量处于布洛赫球上, 这通常被用于描述一个两能级系统。
            \begin{figure}[htbp]
            \begin{center}
              \includegraphics[width= 0.35\columnwidth]{Bloch.eps}
              \caption{量子比特
              }
              \label{bloch-sphere}
            \end{center}
            \end{figure}
        对于一个电子自旋的态$|\psi\rangle$,我们也可以用密度矩阵$\rho$来描述。$\rho$可以被写成自旋泡利矩阵和单位矩阵的线性组合。自旋泡利矩阵可以写成:
            \begin{equation}
            \label{Pauli-matrix}
                \sigma_x = \left(
                             \begin{array}{cc}
                               0 & 1/2 \\
                               1/2 & 0 \\
                             \end{array}
                           \right)~~
                \sigma_y = \left(
                             \begin{array}{cc}
                               0 & -i/2 \\
                               i/2  & 0 \\
                             \end{array}
                           \right)~~
                \sigma_z = \left(
                             \begin{array}{cc}
                               1/2 & 0 \\
                               0 & -1/2 \\
                             \end{array}
                           \right)。
            \end{equation}
        当我们观测电子自旋的时候,如果观测算符为$A$,那么测量期望值$\langle A\rangle$为:
            \begin{equation}
            \label{detection}
            \langle D\rangle_\rho = Tr(D\rho)
            \end{equation}
        当体系在含时哈密顿量$H$支配下进行演化时,我们可以用薛定谔方程对其进行描述:
            \begin{equation}
                \left\{ \begin{aligned}
                            \frac{d}{dt}|\psi\rangle = - i H(t)|\psi\rangle\\
                            \frac{d}{dt}\langle\psi| =  i H(t)\langle\psi|
                        \end{aligned}
                \right.
            \end{equation}
        我们可以把密度算符写成
            \begin{equation}
            \rho(t)= \sum p_k|\psi_k \rangle \langle \psi_k|,
            \end{equation}
        因此可以得到 Liouville-von Neumann 方程\ref{Liouville-Neumann-eqn}。
            \begin{equation}
            \label{Liouville-Neumann-eqn}
                \begin{split}
                    \frac{d\rho(t)}{dt} & = -i\sum p_k H(t)|\psi_k \rangle\langle\psi_k| + i\sum p_k |\psi_k \rangle\langle\psi_k| H(t)\\
                                        & = -i[H(t)\rho(t) -\rho(t) H(t)]\\
                                        & = -i[H(t),\rho(t)]
                \end{split}
            \end{equation}
        当哈密顿量$H$不含时的时候,我们可以得到
            \begin{equation}
            \rho(t)=\exp(-iHt)\rho(0)\exp(iHt),
            \end{equation}
        其中的指数项通常也可以写成
            \begin{equation}
            \label{U-operation}
            U(t)=exp(-iHt)。
            \end{equation}
        $U(t)$被称为传播子或者被称为酉演化,在磁共振量子计算中,我们可以通过脉冲来实现我们所需要的转播子形式\cite{NMR-review},这种传播子被用于实现各种逻辑门操作,成为量子计算的基础操作。

        \subsection{旋转坐标系}
        $S = 1/2$的自旋处于外磁场$B_0$中,假设$B_0$方向沿着$z$轴,式子\ref{E-Zeeman}也可以可以写成
            \begin{equation}
            H_{sys} = \omega_e \sigma_z;
            \end{equation}其中$\omega_e = g \beta_e B_0 $。
        自由演化时的传播子为
            \begin{equation}
            U(t)=exp(-iHt)=\exp(-i\omega_e\sigma_z t)。
            \end{equation}
        这意味着自旋以$z$为转轴,以频率$w_e$旋转。在脉冲式电子顺磁共振实验中,我们施加的方波脉冲可以用$H_p$来描述,$H_p$可以表达为:
            \begin{equation}
            \label{pulse-H}
                \begin{split}
                H_p &= \cos(\omega t+\phi )\gamma_e B_1\sigma_x \\
                    &= \left(
                              \begin{array}{cc}
                               0 & \cos(\omega t+\phi )\gamma_e B_1 \\
                               \cos(\omega t+\phi )\gamma_e B_1 & 0 \\
                              \end{array}
                              \right)
                ;
                \end{split}
            \end{equation}
        这里$\omega$是脉冲的频率,$\phi$是脉冲的相位,$B_1$是脉冲强度。此时描述该体系的哈密顿量$H_{tot} = H_{sys}+H_p$
        我们可以通过一个$\^{U}=\exp(-i\omega\sigma_z t)$变换将实验室坐标系转换到一个绕着$z$轴旋转的坐标系,新坐标系中的哈密顿量$H$可以被写成
            \begin{equation}
            \label{H-rotation1}
            H_{tot}^{\rho} = \left( \begin{array}{cc}
                               (\omega_e-\omega)/2 & \cos(\omega t+\phi )\gamma_e B_1 \exp(-i\omega t) \\
                               \cos(\omega t+\phi )\gamma_e B_1\exp(i\omega t) & -(\omega_e-\omega)/2 \\
                              \end{array}
                    \right)
            \end{equation}
        考虑到$\cos(\omega t + \phi) = (\exp(i(\omega t + \phi))+\exp(-i(\omega t + \phi)))$, 我们可以忽略掉快速的振荡项,有效哈密顿量可以被写成
            \begin{equation}
            \label{H-rotation2}
            H_{tot}^{\rho} =  \left( \begin{array}{cc}
                               \delta  &  \gamma_e B_1 \exp(i\phi) \\
                                \gamma_e B_1\exp(-i\phi) & -\delta \\
                              \end{array}
                    \right)/2,
            \end{equation}
        偏移量$\delta = (\omega_e-\omega)$。

        \subsection{平均哈密顿量理论}
        当高分辨率核磁共振实验技术发展起来的时候,平均哈密顿量理论被发展成为一种有力的分析工具\cite{AHT}。在核磁共振,电子顺磁共振技术被应用于量子计算时,平均哈密顿量理论被用于设计和构造高精度的逻辑门操作\cite{GRAPE-ESR,GRAPE-NMR},或者用于构造想要的哈密顿量。
        这里我们讨论到修正到二阶的平均哈密顿量$H_{eff}$,这在应用中已经有足够的精度。
            \begin{equation}
            H_{eff} = \overline{H}^{(0)} + \overline{H}^{(1)} + \overline{H}^{(2)}
            \end{equation}
            $\overline{H}^{(i)}$($i=0,1,2$)代表不同阶修正量\cite{Schweiger_principles}。
            \begin{equation}
                \begin{split}
                   \overline{H}^{(0)}  &= \frac{1}{t}\int_0^t H(t)dt \\
                   \overline{H}^{(1)}  &= \frac{-i}{2t}\int_0^t dt_1 \int_0^{t_1} dt_2 [H(t_1),H(t_2)]\\
                   \overline{H}^{(2)}  &= \frac{-1}{6t}\int_0^t dt_3 \int_0^{t_3} dt_2 \int_0^{t_2} dt_1 {[H(t_3),[H(t_2),H(t_1)]]+[H(t_1),[H(t_2),H(t_3)]]}
                \end{split}
            \end{equation}
        为了不失一般性,我们接下来讨论磁共振实验中的哈密顿量的一般表达式$H(t) = H_0 + H_{ext}(t)$, $H_0$是样品在磁场中的哈密顿量,它并不含时,$H_{ext}(t)$是外加驱动场的表达式,
            \begin{equation}
                \begin{split}
                    H_{ext}(t)  & = B_1(t)[\cos(\omega t + \phi (t))\sigma_x + \sin(\omega t + \phi(t))\sigma_y] \\
                                & = A(t)\sigma_x + B(t)\sigma_y
                \end{split}
            \end{equation}
        其中 $A(t) = B_1(t)\cos(\omega t + \phi(t))$和$B(t) = B_1(t)\sin(\omega t + \phi(t))$。$A(t)$和$B(t)$可以理解成为沿着正交的两个方向输出的脉冲。通过计算,我们可以得到
            \begin{equation}
            \label{zero-AH}
                \overline{H}^{(0)} = H_0 + \sigma_x\frac{1}{t}\int_0^t A(t)dt + \sigma_y\frac{1}{t}\int_0^t B(t)dt
            \end{equation}
        液体核磁共振实验中常用的成形脉冲一般仅用到零阶平均哈密度量。更高阶的修正(公式\ref{first-AH}和\ref{second-AH})有助于提高成形脉冲搭建的逻辑门操作的保证度。
            \begin{equation}
            \label{first-AH}
                \begin{split}
                    \overline{H}^{(1)}=&-\{  \frac{i}{2t} [\sigma_x, H_0]\int_0^t dt_1 \int_0^{t_1} dt_2 (A(t_1) - A(t_2))\\
                                            &+\frac{i}{2t} [\sigma_y, H_0]\int_0^t dt_1 \int_0^{t_1} dt_2 (B(t_1) - B(t_2))\\
                                            &+\frac{i}{2t} \sigma_z \int_0^t dt_1 \int_0^{t_1} dt_2 (A(t_1)B(t_2) - A(t_2)B(t_1))
                                        \}
                \end{split}
            \end{equation}
            \begin{equation}
            \label{second-AH}
                \begin{split}
                    \overline{H}^{(2)} &=\frac{-1}{6t} \{
                                                         [H_0,[\sigma_x,H_0]]\int_0^t dt_3 \int_0^{t_3} dt_2 \int_0^{t_2} dt_1 (2A(t_2)-A(t_1)-A(t_3))\\
                                                        &+[H_0,[\sigma_y,H_0]]\int_0^t dt_3 \int_0^{t_3} dt_2 \int_0^{t_2} dt_1 (2B(t_2)-B(t_1)-B(t_3))\\
                                                        &+[\sigma_z,H_0]\int_0^t dt_3 \int_0^{t_3} dt_2 \int_0^{t_2} dt_1(A(t_2)B(t_1)-A(t_1)B(t_2)+A(t_2)B(t_3)-A(t_3)B(t_2))\\
                                                        &+[\sigma_x,[\sigma_y,H_0]]\int_0^t dt_3 \int_0^{t_3} dt_2 \int_0^{t_2} dt_1(A(t_3)(B(t_2)-B(t_1))+ A(t_1)(B(t_2)-B(t_3)))\\
                                                        &+[\sigma_y,[\sigma_x,H_0]]\int_0^t dt_3 \int_0^{t_3} dt_2 \int_0^{t_2} dt_1(B(t_3)(A(t_2)-A(t_1))+ B(t_1)(A(t_2)-A(t_3)))\\
                                                        &+\sigma_x \int_0^t dt_3 \int_0^{t_3} dt_2 \int_0^{t_2} dt_1 (B(t_3)(A(t_2)B(t_1)-A(t_1)B(t_2))+B(t_1)(A(t_2)B(t_3)-A(t_3)B(t_2)))\\
                                                        &+\sigma_y \int_0^t dt_3 \int_0^{t_3} dt_2 \int_0^{t_2} dt_1 (A(t_3)(A(t_1)B(t_2)-A(t_2)B(t_1))+A(t_1)(A(t_3)B(t_2)-A(t_2)B(t_3)))
                                                        \}
                \end{split}
            \end{equation}
        真实体系中,方波脉冲并不完美,会受到硬件的限制而产生形变\cite{Schweiger_principles},这时候我们可以用修正到二阶的平均哈密顿量给出有效哈密顿量。我们描述一个沿着$x$轴施加的真实方波脉冲
        $H_{ext}=B_1(t)\sigma_x$为\cite{Hornak1986501}
            \begin{equation}
            \label{real-pulase-shape}
            B_1(t)=\left\{
                     \begin{array}{ll}
                      B_{max}(1-\exp(-t/\tau)),(0<t\leq tp) \\
                      B(tp)\exp(-(t-tp)/\tau), (t>tp)
                     \end{array}
                   \right.
            \end{equation}
        此方程用于描述脉冲的幅度随时间的变化,$B_{max}$是脉冲的最大幅度值,$tp$是理想的方波脉冲长度,$\tau = Q/(\pi \nu_{MW})$是在一定的微波频率$\nu_{MW}$时,由共振腔的品质因子决定,
        可以描述方波脉冲的形变程度,这里我们假定$t\geq tp$。根据上面的讨论,我们可以得出
            \begin{equation}
            \label{real-pulse-AH}
                \begin{split}
                   \overline{H}^{(0)}   =& H_0 + \frac{1}{t}\sigma_x B_{max}(tp+(1-e^{-\frac{t-tp}{\tau}})\tau) \\
                   \overline{H}^{(1)}   =& -\frac{iB_{max}}{2tp}[\sigma_x,H_0]e^{-\frac{t}{\tau}}[e^{\frac{t}{\tau}}tp(t-tp-2\tau)-\tau(t+2\tau)+e^{\frac{tp}{t}}\tau(t+2\tau)]\\
                   \overline{H}^{(2)}   =& \frac{B_{max}}{12t}[H_0,[\sigma_x,H_0]]e^{\frac{-t}{\tau}}(\tau(t^2+6t\tau+12\tau^2)(1-e^{-\frac{tp}{\tau}})\\
                                         & +e^{\frac{t}{\tau}}tp(t^2-3t(tp+2\tau)+2(tp^2+3tp\tau+6\tau^2)))
                \end{split}
            \end{equation}
        当$\tau$接近$0$,也就是脉冲几乎没有形变的时候,上面的式子就简化成为
            \begin{equation}
                \begin{split}
                   \overline{H}^{(0)}   =& H_0 + \frac{tp}{t}\sigma_x B_{max} \\
                   \overline{H}^{(1)}   =& -\frac{iB_{max}}{2}(t-tp)[\sigma_x, H_0]\\
                   \overline{H}^{(2)}   =& \frac{B_{max}}{12t}tp(t-tp)(t-2tp)[H_0,[\sigma_x,H_0]]
                \end{split}
            \end{equation}
        当$t$接近$tp$的时候,那就简化成为理想的方波脉冲的情况。



