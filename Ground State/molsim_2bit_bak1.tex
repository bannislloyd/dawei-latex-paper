\documentclass[twocolumn,showpacs,twoside,10pt,superscriptaddress,prl]{revtex4}
%\documentclass[showpacs,twoside,10pt,superscriptaddress,prl]{revtex4}

\usepackage{graphicx}
\usepackage{epsfig}
\usepackage{epsf}
\usepackage{amssymb}
%\usepackage{epstopdf}
\usepackage{amsmath}
\usepackage{amsthm}
%\usepackage[colorlinks=true,dvipdfm]{hyperref}

\newcommand{\bra}[1]{\langle #1|}
\newcommand{\ket}[1]{|#1\rangle}

%opening


\begin{document}

\title{??Quantum simulation of Molecule Energies}


\begin{abstract}
The difficulty to compute the energy of molecules scales
exponentially whlile quantum algorithm may slove this problem
polynomially. We made one of this kind of this algorithms in an NMR
quantum computer to calculate the energy of H2 molecule's ground
state. The ground state is mapped to the qubits and the energy is
measured by an estimate of the phase which is generated by the
Hamiltonian of the molecule. We utilize the feature of NMR platform
to reduce the difficulty of measuring the phase. And this approach
could be extended to have an arbitrary precision by iterating the
process properly.
\end{abstract}
\maketitle

Quantum computer have been shown in principle more powerful than
classical computers in many problems. The most famous example is the
factoring problem, which has a polynomial complexity in the Shor's
quantum algorithm while no efficient classical method is available.
Another type of classically hard problem that quantum computers
could solve more efficiently is the simulation of other quantum
systems. On classical computers, simulation of $Sch\ddot{o}dinger$
equation of a quantum system need an exponentially growing resource
with the size of the system. In 1982, Feynman proposed that quantum
systems may be well simulated by a quantum computer more efficiently
and then in 1995 S. Lloyd proved this idea to be correct. From this
idea, a few demonstrated examples were shown to simulated the small
quantum systems such as oscillators, Many-Body Fermi
Systems,Heisenberg spin chain and \emph{etc.}.



[need a citation of the importance of molecule energy]Molecular
electron energy is calculated with the  'ab initio' methods. In the
'ab initio' methods, such as the full configuration interaction
(FCI), the dimension of the Hamiltonian matrix increases
exponentially as the molecule becomes larger, which limits the
methods to small molecules such as $H_2O$. However in 2005,
Aspuru-Guzik \emph{etc.} noticed that the quantum phase estimate
(PEA) algorithm could help solve this problem on a quantum
simulator. The PEA algorithm need a polynomial time to determine the
phase of the state, thus after mapping the molecule's space to
qubits' space the energy could be calculated with a polynomial
complexity.

In this letter, we apply the idea of Aspuru-Guzik \emph{etc} to
simulated a $H_2$ molecule on a NMR quantum computer and calculate
the energy of its ground state. The experimental result matches well
with theoretical expectation. And this application could be achieve
an arbitrary precision by iterate the process more times.

\begin{figure}[htb]
\begin{center}
\includegraphics[width= 0.99\columnwidth]{demo_circuit1}
\end{center}
\begin{center}
\includegraphics[width= 0.99\columnwidth]{demo_circuit2}
\end{center}
\label{circuit1}
\end{figure}


The idea of Aspuru-Guzik \emph{etc} is shown in Fig.1a. The quantum
simulator is prepared to $|\Psi\rangle$  the eigenstate of $H_2$
molecule. An unitary operator $U=e^{iH\tau}$ is applied to the state
to generate only phase shift because $H$ is the Hamiltonian of the
$H_2$ molecule. Here we have
$U|\Psi\rangle=e^{iH\tau}|\Psi\rangle=e^{i2\pi\phi}|\Psi\rangle$
where $E=2\pi\phi/\tau$ is the energy corresponding to the eigen
state $|\Psi\rangle$. Then they adopted a standard quantum phase
estimate algorithm to measure the phase shift generated by the
operator to get the energy value. Noted that, $\tau$ is chosen
properly to make the phase $\phi$ ranges from 0 to 1. So after the
$k-th$ loop to measure the phase $\phi_k$, they choose another
operator $U_{k+1} =(e^{-i2\pi\phi_k})U_k$ to generate more bits of
the phase value ,thus the energy could in principle be arbitrarily
accurate only if the operator $U_k$ is available.

The most straightforward way to measure the phase shift is shown in
Fig1b. For the first iteration, after the controlled-U operator, the
ancillary qubit is placed on state $\frac{1}{\sqrt{2}}(|0\rangle +
e^{i2\pi\phi}|1\rangle)$. And right before measurement, the
ancillary qubit is $\frac{1}{2}[(1+e^{i2\pi\phi})|0\rangle +
(1-e^{i2\pi\phi})|1\rangle]$. So in the measurement, it takes the
probility $P_0=\cos^2{(\pi\phi)}$ to measure "0" and
$P_1=\sin^2{(\pi\phi)}$ to measure "1". By repeating this process
$N$ times, $P_0$ and $P_1$ can be determined to an accuracy of
$\frac{1}{\sqrt{N}}$. Thus this method need exponentially
$N\sim2^{2l}$ rounds to get $l$ accurate bits of $\phi$. However in
the NMR quantum computer which we are engaging, the objective
"qubit" we are dealing with is a ensemble of single nuclear spins,
so when we do a "measurement" actually we measure
$10^{XXXIdon'tknow}$ copies of the qubits and get the distribution
of all values(0 or 1). This ensures that we could get almost
$log_210^{XXXIdon'tknow}$ bits of the phase in each measurement.
Thus we could take this advantage in our demonstrated example for
each iteration to get the value of $\phi_k$ and calculate the energy
of the molecule.

Because of the limitation in quantum computation technologies, to
calculate a large molecule's energy in not possible for the time
being. So we choose to calculate the ground energy of $H_2$ molecule
for our demonstration.

The (???internal) Hamiltonian of $H_2$ molecular is shown as
follows,
$$
    H=\Sigma_i(T_i+\Sigma_j{V_{ij}})+\Sigma_{i>j}{O_{ij}}
$$
where $T_i$ is the kinetic energy of electron i, and $V_ij$ is the
coulomb potential energy between the $i$th electron and the $j$th
nucleus, while $O_ij$ is the coulomb potential energy between
electrons $i$ and $j$. Here we perform the simplest situation: $H_2$
in the minimal STO-3G basis. In this two-nucleus and two-electron
molecule, each atom has a $1s$ Guassian-type function, and the two
functions compose one abonding orbital with gerade symmetry and one
antibonding orbital with ungerade symmetry. So there are 4 spin
orbitals and these 4 spin orbitals can form 6 configurations.
Considering the singlet symmetry and the spatial symmetry of $H_2$
exact ground state, only two configurations are acting in fact in
the calculation: the ground state configuration $|\Psi_0\rangle$ and
the double excitation configuration
$|\Psi_{1\bar{1}}^{2\bar{2}}\rangle$. Thus, the Hamiltonian matrix
is(in atom units, the nucleus distance is 1.4\emph{a.u.}, only the
electron's energy) : [2]

\begin{eqnarray}
        H_{pro}&=&
        \begin{pmatrix}
            \langle\Psi_0|H|\Psi_0\rangle & \langle\Psi_{1\bar{1}}^{2\bar{2}}|H||\Psi_{1\bar{1}}^{2\bar{2}}\rangle\\
            \langle\Psi_{1\bar{1}}^{2\bar{2}}|H|\Psi_0\rangle & \langle\Psi_{1\bar{1}}^{2\bar{2}}|H|\Psi_{1\bar{1}}^{2\bar{2}}\rangle
            \end{pmatrix}\nonumber\\
        &=& \begin{pmatrix}\nonumber\\
            -1.8310 & 0.1813\\
            0.1813 & -0.2537
            \end{pmatrix}\\
\end{eqnarray}
whose eigenvalue is -1.8516 \emph{a.u.} .
\end{document}
