\documentclass[onecolumn,superscriptaddress,10pt,showpacs,showkeys,pla]{revtex4}%
\renewcommand\baselinestretch{2}
\usepackage{mathrsfs}%
\usepackage{natbib}
\usepackage{bm}
\usepackage{amsmath}
\usepackage{threeparttable}
\usepackage{amssymb}
\usepackage{amsfonts}
\usepackage[colorlinks,CJKbookmarks,linkcolor=blue]{hyperref}
\usepackage{color}
\usepackage{booktabs}  %���岻ͬ��ϸ�ķָ���
\usepackage{tabularx}  %������
\usepackage{graphicx}%

\begin{document}

\subsection*{Abstract}

Using NMR techniques, we simulate the violations of two Bell-type
inequalities: Mermin-Ardehali-Belinskii-Klyshko (MABK) inequality
and Chen's inequality, for the 3-qubit generalized GHZ states. The
experimental results are in good agreement with the quantum
predictions and show that Chen's inequality is more efficient than
MABK inequality in the case of the generalized GHZ entangled states.





\subsection*{Conclusions}

In summary, we have investigated the simulation of the violation of
Bell-type inequalities, including MABK inequality and Chen's
inequality for the generalized GHZ states in an NMR system. In the
range of the generalized GHZ states, Chen's inequality is more
efficient than MABK inequality. The experimental results are well in
agreement with the expectation of quantum mechanics.

It is necessary to emphasize that, in strict, because NMR qubits are
many nuclear spins of atoms bounded together in a single molecule,
separated by a few angstroms, the NMR experiment is inherently
local. Whereas, the meaning is that, when we experimentally simulate
the violation of different Bell-type inequalities for arbitrary
generalized three-qubit GHZ states in NMR, the results are
excellently in accord with the quantum predictions. It tells us,
despite of many existed disputes, NMR may contribute more on some
fundamentals of quantum mechanics. As a refined tool and technique
for experimentally realizing quantum computation in the last decade,
NMR is still contributing to numerous fundamental problems of
quantum mechanics now. In the future, we will still pay attention to
this area.


\end{document}
